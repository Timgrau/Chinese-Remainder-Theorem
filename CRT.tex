\documentclass{scrreprt}

\usepackage[utf8]{inputenc}
\usepackage[T1]{fontenc}
\usepackage[ngerman]{babel}
\usepackage{blindtext, amsmath, amssymb, amsthm}

\newtheorem{definition}{Definition}[section]
\newtheorem{lemma}[definition]{Lemma}
\newtheorem{verk}[definition]{Verknüpfungstabelle}

\theoremstyle{remark}
\newtheorem{beispiel}[definition]{Beispiel}

\title{Chinese Remainder Theorem}
\author{Timo Grautstück}
\date{18. Juni 2020}

\begin{document}
\maketitle
\tableofcontents

\chapter{Einleitung}
Das ``Chinese Remainder Theorem'' im Deutschen unter anderem als chinesischer Restklassensatz bezeichnet, ist ein Theorem der abstrakten Algebra und Zahlentheorie. Die erste Schrift des Theorems stammt von dem chinesischen Mathematiker Sun Zi aus seinem damaligen Buch ``Sun Zis Handbuch der Arithmetik'' ca. 3. Jahrhundert.
Man nutzt das Theorem unter anderem, um großzahlige Modulorechnungen  mittels mehrerer kleinzahliger Berechnungen zu bestimmen oder zur Berechnung simultaner Kongruenzen.

\chapter{Kongruenzen und Restklassen}
Der Chinesische Restklassensatz funktioniert, indem man eine Rechenoperation Namens Modulo nutzt \textit{(Division mit Rest)}. Um das ganze besser zu verstehen, sollte man einige grundlegende Begriffe klären, wie Kongruenz oder Restklassen.

\section{Kongruenzen}
\begin{definition}
  Ist $m\in\mathbb{N}$ und $a,b\in\mathbb{Z}$ kann man sagen, dass $a$ kongruent zu $b$ modulo $m$ ist, wenn $m$ die Differenz $b-a$ teilt.
\end{definition}

\[
\boxed{a\equiv b\mod m}
\]

\begin{beispiel}
\end{beispiel}
\begin{center}
  $10 \equiv 0 \mod 5$ \,\,\,,\,\,\, $12 \equiv 2 \mod 5$
\end{center}\\
\hrulefill\\\\
Teilt $m$ jedoch $b-a$ nicht, so bezeichnet man $a$ inkongruent zu $b$ modulo $m$. Um die Definition zu unterstützen, würde ich den Vorgang der modulo Operation wie folgt berschreiben:\\
Sei eine Zahl $22$ gegeben und diese entspricht $a$. Soll nun $a$ modulo $7$ berechnet werden, entspricht $m = 7$. Nun ist die Frage welche Zahl entspricht $b$ bzw. $22$ ist kongruent welcher Zahl modulo $7$.

\[
\boxed{22\equiv b \mod 7}
\]

\begin{lemma}
  \hfill
  \begin{enumerate}
  \item Wie oft passt $m$ in $a$ ?  
  \item Wenn $m$, $a$ nicht ganzzahlig teilt, welcher Rest bleibt über ?
  \item Falls ein Rest existiert, ist dieser Rest $b$, existiert kein Rest entspricht $b = 0$.
  \end{enumerate}
\end{lemma}

\begin{beispiel}
  \hfill
  \begin{enumerate}
    \item $3\cdot7 = 21$
    \item $22-21 = 1$
    \item $22 \equiv 1 \mod 7$
  \end{enumerate}
  \end{beispiel}
Man bezeichnet 22 als Representant der Restklasse $\left\bar1$ auf $\mathbb{Z}_{7}$,  genau wie $\lbrace\ldots,-8,1,8,14,\ldots\rbrace$

\section{Restklassen}

Da jetzt klar ist, dass man sich bei Modulo für Reste interessiert, kann man nun auf Restklassen schauen. Wenn man von dem vorherigen Beispiel ausgeht, dass $m$ $7$ entspricht, dann sei die Menge von $\mathbb{Z}_{7}:=\lbrace0,1,2,3,4,5,6\rbrace$. Diese Menge entspricht auch den Restklassen von $\mathbb{Z}_{7}$. Um Restklassen darzustellen, wird in diesem Dokument folgende Notation $\bar b$ genutzt. Restklassen enthalten unendlich viele Elemente, jedoch enthalten zwei unterschiedliche Restklassen von einem bestimmten Modul $m$, niemals die selben Elemente. Jedoch ihr eigenes Element.
\begin{definition}
  Sei $m\in\mathbb{N}$ und $b\in\mathbb{Z}$, dann bezeichnet man die folgende Menge $\bar b:=\lbrace a\in\mathbb{Z}\,\vert\, a\equiv b \mod m\rbrace$ als Restklasse von $\mathbb{Z}_m$ und $\forall a \in\bar b$ als Representant von $\bar b$.
\end{definition}

\begin{beispiel}
\end{beispiel}
\[
\boxed{\bar0 := \lbrace\ldots ,-21,-14,-7, 0, 7, 14, 21,\ldots\rbrace}
\]
\section{Veranschaulichung von Restklassen}
\[
  \boxed{
    \begin{tabular}{c|c|c|c|c|c|c}
      -21 & -20 & -19 & -18 & -17 & -16 & -15\\
      -14 & -13 & -12 & -11 & -10 & -9 & -8\\
      -7 & -6 & -5 & -4 & -3 & -2 & -1\\
      $\bar0$ & $\bar1$ & $\bar2$ & $\bar3$ & $\bar4$ & $\bar5$ & $\bar6$\\
      7 & 8 & 9 & 10 & 11 & 12 & 13\\
      14 & 15 & 16 & 17 & 18 & 19 & 20\\
      21 & 22 & 23 & 24 & 25 & 26 & 27\\
    \end{tabular}
  }
  \]\\\\
Nun kann man vertikal die jeweilig enthaltenden Elemente \textit{Representanten} der unterschiedlichen Restklassen erkennen.
Wenn man das nun mit der vorherigen Definition 2.1.1 prüft, sollte gelten, dass $\forall a\in\bar b$, $m$ Teiler von $b-a$ ist.
\begin{proof}
  $\rightArrow a = -21$\,,\,\,\,$b = 0$\,,\,\,\,$m = 7$
  \begin{align}
    &\, -21 \equiv 0 \mod 7\\
    &= 0-(-21) = 21\\
    &= 21/7 = 3 \qedhere
  \end{align}
\end{proof}
\hrulefill\\\\
Man kann 21 durch 7 ganzzahlig teilen, also gilt auch 2.1.1 und wie in 2.1.3 zu sehen, dass $b = 0$ sein muss. Man nennt die Menge aller Restklassen von $\mathbb{Z}_m$,\rightarrow\,\,$R_m$.\\ $R_m\neq\mathbb{Z}_m$, da $R_m$ die Mengen der Restklassen enthält und $\mathbb{Z}_m$ die Vereinigung der Mengen der Restklassen.
%---------------------------------------------------------------------------------------------------------------------------------------------------------------------------------------------------------------------------%
\chapter{Ringe und Körper}
Ringe und Körper sind algebraische Strukturen. Mengen $G$ bilden diese Strukturen durch Einhalten von verschiedenen Axiomen mit mathematischen Operationen $\circ$\,\rightarrow\,$(\oplus,\otimes)$.\\
\[
\boxed{(G,\,\¸\circ\,\,)}
\]
\section{Axiome}

\begin{lemma}
  \begin{enumerate}
    \hfill
  \item Abgeschlossenheit: \,\,\,$\left[x\circ y\in G\right]$
  \item Assoziativität:\,\,\,$(x\circ y)\circ z = x\circ (y\circ z)$
  \item Neutrales Element:\,\,\,$[e\in G] : e\circ x = x$
  \item Inverses Element:\,\,\,$[\exists x^{-1}\in G]:\,\,\forall x\in G: x\circ x^{-1} = e$
  \item Kommutativität:\,\,\,$x\circ y = y\circ x$
  \end{enumerate}
\end{lemma}

\section{algebraische Strukturen}

\begin{itemize}
\item 1, 2 $\rightarrow$ Halbgruppe
\item 1, 2, 3 $\rightarrow$ Monoid (Halbgruppe mit 1.)
\item 1, 2, 3, 4 $\rightarrow$ Gruppe
\item 1, 2, 3, 4, 5 $\rightarrow$ Kommutativegruppe
\end{itemize}

\section{Ringe}
\begin{definition}
Eine Menge $(G,\¸\,\circ\,\,)$ nennt man Ring, wenn die folgenden Axiome mittels Multiplikation und Addition erfüllt werden:
    \begin{enumerate}
    \item $(G,\,\oplus\,)$ ist kommutative Gruppe mit neutralem Element $0$.
    \item $(G,\,\otimes\,)$ ist Halbgruppe.
    \item Distributivität: $x\cdot(y+z) = x\cdot y+x\cdot z$
    \end{enumerate} 
\end{definition}

\section{Körper}
\begin{definition}
  Ein Körper ist ein kommutativer Ring mit Einselement, in dem jedes von Null verschiedene Element invertierbar ist. Es müssen folgende Axiome erfüllt werden:
  \begin{enumerate}
  \item $(G\,,\oplus\,)$ ist kommutative Gruppe mit neuralem Element $0$.
  \item $(G\setminus\lbrace0\rbrace\,,\otimes\,)$ ist kommutative Gruppe mit neutralem Element $1$.
  \item Distributivität: $x\cdot(y+z) = x\cdot y+x\cdot z$
  \end{enumerate}
\end{definition}
\hrulefill\\\\

\section{Restklassenring}
Ein Restklassenring ist ein Faktorring, der aus Restklassen besteht.
\subsection{Verknüpfungstabelle (Addition)}
Durch Verknüpfungstabellen, kann man sehr gut veranschaulichen, welche Restklassen existieren und was es für neutrale bzw. inverse Elemente gibt.\\\\
\begin{minipage}{0.7\textwidth}
  \begin{tabular}{|c||c|c|c|c|}
    \hline
    $(\mathbb{Z}_{4},\oplus)$& 0 & 1 & 2 & 3 \\
    \hline
    \hline
    0 & 0 & 1 & 2 & 3 \\
    1 & 1 & 2 & 3 & 0 \\
    2 & 2 & 3 & 0 & 1 \\
    3 & 3 & 0 & 1 & 2\\
    \hline 
  \end{tabular}
\end{minipage}
\begin{minipage}{0.7\textwidth}
  \begin{tabular}{|c||c|c|c|c|c|}
    \hline
    $(\mathbb{Z}_{5},\oplus)$& 0 & 1 & 2 & 3 & 4\\
    \hline
    \hline
    0 & 0 & 1 & 2 & 3 & 4 \\
    1 & 1 & 2 & 3 & 4 & 0 \\
    2 & 2 & 3 & 4 & 0 & 1 \\
    3 & 3 & 4 & 0 & 1 & 2 \\
    4 & 4 & 0 & 1 & 1 & 3 \\
    \hline 
  \end{tabular}
\end{minipage}\\
\\\\
Nun erkennt man, dass das Neutrale Element, wie im Unterpunkt 3. \textit{(Axiome 3.1)} beschrieben, für die Addition auf dem Modul $\mathbb{Z}}_{4}$ die Null ist.
  \begin{proof}
    \hfill
    \begin{center}
      $\,0\in\mathbb{Z}_{4}: 0+1\equiv1 \mod 4$\\
      $&\,0\in\mathbb{Z}_{5}: 0+2\equiv2 \mod 5$
    \end{center}
  \end{proof}
  
Da ein neutrales Element besteht, könnte man prüfen ob auch ein inverses Element besteht. Um ein inverses Element von einer bestimmten Zahl oder auch Repräsentanten zu bestimmen, sollte diese Zahl mit ihrem inversen Element durch die jeweiligen Rechenvoschrift \textit{Addition, Multiplikation} wieder das neutrale Element ergeben. Auch dafür kann der ``Chinesische Restklassensatz'' angewendet werden. Jedoch kann man dies auch gut in der \textit{Verknüpfungstabelle 3.5.1} ablesen.
\begin{proof}
  \hfill
  \begin{center}
    $ \left[3 \in\mathbb{Z}_{4}\right]:1 \in\mathbb{Z}_{4}: 1+3\equiv0 \mod 4$ \\
     $ \left[2 \in\mathbb{Z}_{5}\right]:3 \in\mathbb{Z}_{5}: 3+2\equiv 0 \mod 5$
    \end{center}
\end{proof}
Man hat durch die Verknüpfungstabelle herausgefunden, dass die in \textit{3.1 Axiome} 3. \& 4. für $(\mathbb{Z}_{4},\,\oplus\,)$ sowie $(\mathbb{Z}_{5},\,\oplus\,)$ zustimmen. Auch (1.; 2.; 5.) erfüllen diese beiden Strukturen. Daher kann man laut \textit{3.2 algebraische Strukturen} davon ausehen, dass es sich hier um eine Kommunikativegruppen handelt.

\subsection{Verknüpfungstabelle (Multiplikation)}
\begin{minipage}{0.7\textwidth}
  \begin{tabular}{|c||c|c|c|c|}
    \hline
    $(\mathbb{Z}_{4},\otimes)$& 0 & 1 & 2 & 3 \\
    \hline
    \hline
    0 & 0 & 0 & 0 & 0 \\
    1 & 0 & 1 & 2 & 3 \\
    2 & 0 & 2 & 0 & 2 \\
    3 & 0 & 3 & 2 & 1\\
    \hline
  \end{tabular}
\end{minipage}
\begin{minipage}{0.3\textwidth}
  \begin{tabular}{|c||c|c|c|c|c|}
    \hline
    $(\mathbb{Z}_{5},\otimes)$& 0 & 1 & 2 & 3 & 4\\
    \hline
    \hline
    0 & 0 & 0 & 0 & 0 & 0 \\
    1 & 0 & 1 & 2 & 3 & 4 \\
    2 & 0 & 2 & 4 & 1 & 3 \\
    3 & 0 & 3 & 1 & 4 & 2 \\
    4 & 0 & 4 & 3 & 2 & 1 \\
    \hline 
  \end{tabular}
\end{minipage}
\\\\
Es fällt bei der Verknüpfungstabelle der Multiplikation eine Besonderheit auf. Auch hier gibt es ein inverses Element, das wäre in diesem Fall auf der Struktur $(\mathbb{Z}_{n},\,\otimes\,)$ die $1$.
\begin{proof}
  \hfill
  \begin{center}
       $\,1\in\mathbb{Z}_{4}: 1\cdot3\equiv3 \mod 4$\\
      $&\,1\in\mathbb{Z}_{5}: 1\cdot2\equiv2 \mod 5$
  \end{center}
\end{proof}
Jedoch existiert nicht für jedes Element eine multiplikative Inverse. Denn solange man auf die Struktur $(\mathbb{Z}_{m},\,\otimes\,)$ schaut und $m$ keine Primzahl ist. Dann existiert für jedes Element das den selben größten gemeinsamen Teiler \textit{([gcd(a,b)] siehe. Euclid's algorithm)} wie $m$ hat, keine multiplikative Inverse. (Verknüpfungstabelle $\mathbb{Z}_{4}$ [2,2])
\begin{proof}
  \hfill
  \begin{center}
    $2\equiv2 \mod4$\\
    $4\equiv0 \mod4$\\
    $10\equiv2 \mod4$\\
  \end{center}
\end{proof}
Daraus kann man schließen, wenn $n$ keine Primzahl ist. Wird auf der Struktur $(\mathbb{Z}_{n},\,\otimes\,)$, das \textit{4.Axiom} nicht erfüllt, jedoch (1.; 2.; 3.; 5.). Also erhält man für die Struktur $(\mathbb{Z}_{n},\,\otimes\,)$ einen Monoid.\\
Damit ist $(\mathbb{Z}_{n},\oplus,\otimes)$ ein Ring (\textit{3.3 Ringe}), da ein Monoid die Kriterien einer Halbgruppe erfüllt und 3. die Distributivität gegeben ist. So erhält man für die Restklassen einen Restklassenring.\\
\hrulefill
Wenn $m$ jedoch eine Primzahl ist, dann finden man eine multiplikative Inverse für jedes Element, außer dem Nullelement. \textit{Siehe 3.5.2 ($\mathbb{Z}_{5},\,\otimes\,$)}.\\
Das heißt diesmal wird das \textit{3. Axiom} erfüllt und wir erhalten für die Struktur $(\mathbb{Z}_{m},\oplus,\otimes)$ einen Körper. Somit bezeichnet man die Restklassen als Restklassenkörper.
%-----------------------------------------------------------------------------------------------------------------------------------------------------------------------------------------------------------------------------
\chapter{Chinese Remainder Theorem}
\section{großzahlige Moduloberechnungen}
Wenn $x\equiv14 \mod 84$ gegeben ist und x gesucht wird, kann man dies mithilfe des Chinesischen Restsatzes.\\
\begin{equation}
  \[
  \boxed{
    \sum_{i} [M_{i}]_{m}\cdot[M_{i}]_{m_{i}}^{-1}\cdot x_{i}\mod{m}
  }
  \]
\end{equation}
Man geht wie folgt vor:
\begin{lemma}
  \begin{enumerate}
    \hfill
  \item Zerlege $m = 84$ in $m_{i}$ Faktoren. 
  \item Zerlege $x$ in $x_{i}$ durch $x \equiv x_{i}\mod{m_{i}}$
  \item Berechne $M_{i}\mod{m_{i}}$ durch z.b $m_{2}\cdot m_{3}\equiv [M_{1}]\mod{m_{1}}$
  \item Ermittel die multiplikative Inverse $M_{i}^{-1}$ durch $M_{i}\cdot M_{i}^{-1}\equiv e\mod{m_{i}}$
  \item Berechne $x$ durch $[M_{i}\cdot M_{i}^{-1}\cdot x_{i}]$.
  \end{enumerate}
\end{lemma}

\begin{beispiel}
  \begin{enumerate}
    \hfill
  \item $84 = 3\cdot4\cdot7 = m_{1}\cdot m_{2} \cdot m_{3}$
  \item
    \begin{itemize}
    \item $14 \equiv 2 \mod 3\rightarrow [2 = x_{1}]$
    \item $14 \equiv 2 \mod 4\rightarrow [2 = x_{2}]$
    \item $14 \equiv 0 \mod 7\rightarrow [0 = x_{3}]$
    \end{itemize}
  \item
    \begin{itemize}
    \item $4\cdot 7\equiv 1 \mod{3} \rightarrow [M_{1}\mod{m_{1}}]$
    \item $3\cdot 7\equiv 1 \mod{4} \rightarrow [M_{2}\mod{m_{2}}]$
    \item $3\cdot 4\equiv 5 \mod{7} \rightarrow [M_{3}\mod{m_{3}}]$
    \end{itemize}
  \item
    \begin{itemize}
      \item $1\cdot1\equiv 1 \mod{3} \rightarrow [M_{1}^{-1} = 1]$ 
      \item $1\cdot1\equiv 1 \mod{4} \rightarrow [M_{2}^{-1} = 1]$
      \item $5\cdot3\equiv 1 \mod{3} \rightarrow [M_{3}^{-1} = 3]$ 
    \end{itemize}
    \item $(4\cdot7\cdot1\cdot2)+(1\cdot3\cdot7\cdot2)+(3\cdot4\cdot3\cdot0) = \boxed{98\equiv14 \mod{84}}$
  \end{enumerate}
\end{beispiel}
%--------------------------------------------------------------------------------------------------------------
\chapter{Euklidischer Algorithmus}
Der Euklidische Algorithmus kann dafür genutzt werden, um den größten gemeinsamen Teiler zu bestimmen \textit{ggT(a,b)}. Jedoch kann der erweiterte Euklidische Algorithmus auch dafür genutzt werden, um die Multiplikative Inverse von $ e \equiv a^{-1}\cdot a \mod{b} $ zu bestimmen. Jedoch kann diese nur gefunden werden, wenn der \textit{ggT(a,b) = 1} ist.

\section{Algorithmus}
Um den größten gemeinsamen Teiler von \textit{ggT(a,b)} zu bestimmen, gehe man wie folgt vor, schreibe untereinander:
\begin{enumerate}
  \item Wie oft passt a ganzzahlig in b ? \rightarrow \,\,$\!b = x\,\cdot\,a$ 
  \item entsteht ein Restm, dann Addiere \rightarrow \,\,$\!b = x\,\cdot\, a\,\,\,+ r$
  \item Ist der Rest = 0, dann ist der ggT(a,b) der vorherige Rest. 
\end{enumerate}
\section{erweiterter Algorithmus}

\end{document}
